\chapter{Conclusion}
\label{cha:conclusion}

This work presents an analysis of the practicality of monitoring health parameters of walker users by sending data over LoRWAN. 

Our experiments indicate that if the node is optimized for its energy consumption, which we didn't have time to do, the LoRaWAN protocol, especially when integrated with a stack like \textit{The Things Network}, provides a very simple and streamlined way of connecting walkers that can send their information over very large distances, with high reliability and serviceability. 

We belive that with higher quality sensors, well integrated into the walker, the overall fidelity of the system would be very high, making it a viable way to monitor health parameters overtime, since latency is not an issue in that use case.

As for future work, having the arduino being capable of using LoRaWAN outside and switching to 



%%% Local Variables:
%%% mode: latex
%%% TeX-master: "../ClassicThesis"
%%% End:
