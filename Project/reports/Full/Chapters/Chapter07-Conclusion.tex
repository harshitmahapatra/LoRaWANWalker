\chapter{Conclusion}
\label{cha:conclusion}

This work presents an analysis of the practicality of monitoring health parameters of walker users by sending data over LoRWAN. 

We attached an accelerometer, a pulse sensor, a GPS sensor and two pressure sensors to a walker, and had an arduino collecting and sending their data over LoRaWAN, so it could be stored on a remote server for future analysis.

Our experiments indicate that if the node is optimized for its energy consumption, which we didn't have time to do, the LoRaWAN protocol, especially when integrated with a stack like \textit{The Things Network}, provides a very simple and streamlined way of connecting walkers that can send their information over very large distances, with high reliability and serviceability. 

We belive that with higher quality sensors, well integrated into the walker, the overall fidelity of the system would be very high, making it a viable way to obtain health parametets. Our system can achieve a relatively small throughput of 8.4 bytes per minute, but since the goal is not real-time tracking, this is not a concern.

Having all this in consideration, we can conclude that with further improvements to overall quality and battery life, sending walker data over LoRaWAN could be a practical way to monitor long term health parameters.

As for future work, having a node capable of storing and pre-processing information would allow for a better adaption to the limitations of LoRaWAN, making the most of the small amount of information that can be sent. Another interesting addition would be the capability of using a GSM enabled module for alerting in emergency scenarios and the ability to send higher bandwidth data, for instance, a video feed.

%%% Local Variables:
%%% mode: latex
%%% TeX-master: "../ClassicThesis"
%%% End:
