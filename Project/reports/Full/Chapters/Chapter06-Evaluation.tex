\chapter{Evaluation}
\label{cha:evaluation}

Having in mind our hypothesis: "Is it practical to monitor walkers and their users by sending data over LoRaWAN?"

The parameters of our system that have to be measured to test practicality were outlined in the introduction and the tests we intended to run were outlined in the "Design and Methodology" chapter.

\section{Feasibility}
\subsubsection{Feasibility}
As a baseline for our 

We expected to be able to build and use the system, this is just a baseline parameter.

We managed to successfully send data from sensors on the walker to our server, so we can say 



\section{Normal Usage}

	\subsection{Pourpose}
	We are running this big experiment because measuring accuracy, consistency, reliability, latency and energy consumption, all have in common the fact that the best way to get data on these parameters is simply using the walker, so we devised this test, which has the goal of helping us simultaneously assess all the aforementioned quality parameters.

	\subsection{Expectations}
		\subsubsection{Consistency}
			We have different expectations, depending on the sensor:

			\begin{enumerate}
				\item Location: Our module's datasheet does not provide a precision threshold, so we don't know what to expect
				\item Heart Rate: The sensor we are using is very sensitive to noise and we have no way of keeping our heart rate constant when running the test, so we expect the readings to vary quite a lot.
				\item Movement: Since this is a boolean value, we expect it to be quite consistent, always reporting movement or no movement, given the same scenario.
				\item Pressure: We expect that if we manage to put constant force on the pressure sensors, they will give consistent readings
			\end{enumerate}

		\subsubsection{Accuracy}
			Again, each sensor can be said to be accurate if it meets conditions specific to it 

			\begin{enumerate}
				\item Location: According to the specification, the NEO 6M module should be accurate within 2 meters, so that is our expectation.
				\item Heart Rate: The sensor we are using is very sensitive to the pressure put on it and the thickness of the person's skin. We are also calculating the user's heart rate by processing the raw pulse data, so we don't expect much accuracy
				\item Movement: We are only using the accelerometer to measure if the walker is moving, so we expect an accuracy of 100\%, since we see no reason it would give an erroneous report
				\item Pressure: The pressure sensor we use does not provide an accuracy threshold, but from our preliminary usage it seems to be at least 95\% accurate
			\end{enumerate}

		\subsection{Reliability}
			The only failure we could see happening in this experiment would be LoRaWAN packets being sent from the node to the gateway being dropped, which doesn't seem to happen very often. We are thus expecting at least 90\% reliability

		\subsection{Latency}
			The air-time of LoRaWAN packets is dependent on the payload, the distance to the gateway, the spreading factor and much more, so it is hard for us to make an accurate prediction. We have nonetheless noticed, while building our system, that the time between packets is usually between 4 and 9 minutes, so we are expecting an average of around 6 minutes and a high standard deviation from the mean.
		\newline
		\newline
		\newline
		\subsection{Energy consumption}
			The power consumption of our components is as follows:

			\begin{table}[h]
				\begin{tabular}{@{}ll@{}}
					\toprule
					Sensor         & Max current draw (mA) \\ \midrule
					Pulse          & 4                     \\
					Pressure (x2)  & 3                     \\
					Accelerometer  & 0.5                   \\
					GPS            & 57                    \\
					LoRaWAN shield & 10                    \\
					Mega 2560      & 80                    \\
					Total          & 159                   \\ \bottomrule
				\end{tabular}
			\end{table}

			Based on this, we expect the current draw to be around 160mA

	\subsection{Parameters}
		We had on of our team's members use the walker normally for 30 minutes, while the information was coming in and stored in our server for analysis. There were only three details:

		\begin{itemize}
		  \item In order to test the accuracy of the heart rate, we had the team member testing the walker use an Iwatch for comparison
		  \item In order to test the consistency of the pressure sensor readings, we put an elastic band applying constant force on the right handle's sensor
		  \item In order to measure the current draw we used a USB Current Meter during the experiment
		\end{itemize}

	\subsection{Results}
		\subsection{Latency}

			Here are the times, in minutes and seconds, between packets:
			\begin{table}[h]
				\begin{tabular}{@{}lll@{}}
					Gap 	    &	First Run   & 	Second run 	 \\
					1st - 2nd 	&	7:35		& 	6:00     	 \\
					2nd - 3rd 	&	8:06    	& 	4:43         \\
					3rd - 4th 	&	6:47		& 	5:05         \\
					4th - 5th 	&	6:13    	& 	5:14         \\
					5th - 6th 	&	         	& 	4:24         \\
				\end{tabular}
			\end{table}

			For the 1st run we got an average of 7:18

		\subsubsection{Heart Rate}
			First run:
			\begin{table}[h]
				\begin{tabular}{@{}ll@{}}
					Our Reading & IWatch Reading \\
					60          & 72             \\
					70          & 76             \\
					00          & 82             \\
					65          & 72             \\
					66          & 74            
				\end{tabular}
			\end{table}
			Second Run:
			\begin{table}[h]
				\begin{tabular}{@{}ll@{}}
					Our Reading & IWatch Reading \\
					72          & 64             \\
					00          & 64             \\
					62          & 72             \\
					65          & 70             \\
					61          & 65      		 \\      
					64          & 64
				\end{tabular}
			\end{table}


	\subsection{Matching results and expectations}
		\subsubsection{Heart Rate}
			In both runs we got a reading of 0 in on of the measures, so we will takes these as errors of the overall system and not take them into account in our statistical analysis.

			In the first run we got a mean average percentage error(MAPS) of 11.27, between this and the second run, we found an improvement that could be done to the algorithm, so we changed it for the second run, where we got a MAPS of 7.94. Given the constraints mentioned in our expectations, these results are better than we expected, perhaps because %34r come up with explanation

			We feel like doing a precision analysis is not appropriate, since there was a gap of several minutes between measurements and the person was walking around, so there is no reason we should expect the heart rate to keep constant.


\section{Idle Energy Consumption}

	\subsection{Pourpose}
		Most of the time, the walker will not be in use, so it is important to know how much energy it draws while idle
	\subsection{Expectations}

	\subsection{Parameters}
		We connected a USB Current Meter to the walker and left it idle for 5 minutes, while monitoring the current it was drawing
	\subsection{Results}
		The walker drew 0.2A throughout the 5 minutes, which
	\subsection{Matching results and expectations}











\section{Integrating the GPS sensor}
	\subsection{Pourpose}
	This experiment has the goal of figuring out how quickly a new sensor can be added to the walker. Our walker is a proof of concept for something bigger, so it is important to know how extensible it is.

	\subsection{Expectations}
	Having in consideration how much time it took us to mount, get readings, and send the current sensor's information, we expect to take around 1 or 2 days to integrate the GPS sensor into the system. 

	\subsection{Parameters}
	For this experiment, we had two of the team's members work exclusively on integrating the GPS sensor into the system. This consisted in mounting it on the walker, getting some readings, figuring out the best way to encode the data for sending over LoRaWAN, and then adapting the rest of the components to handle the latitude and longitude.

	\subsection{Results}
	Integrating GPS into the system ended up taking 2 whole days, not only was it hard to find a port layout which allowed all the sensors to work, but library support is also not very good, so after unsuccessfully trying to use external libraries, we ended up implementing it ourselves, which took some extra time.

	\subsection{Matching results and expectations}
	Integrating GPS into our system took as much time as we were roughly expecting, perhaps it is on the higher end of our expectation because of the usual delays that come with getting the right documentation for sensors and how they interplay with the other components. While performing this experiment, coincidentally, we noticed that when the GPS wasn't working and we wanted to use the walker for other tests, we had to remove it, so we could also say the system is modular in the sense that it is easy to remove sensors.



\section{Setting up the Arduino from scratch}

	\subsection{Pourpose}
		With this experiment we can get an idea of how scalable our system is, which is one of its most important features.
	\subsection{Expectations}
		We expect the first atempt to take around 4 minutes because we haven't registered a device in a while, but if the process is streamlined we expect it to take around 1:30 minutes, since it just requires navigating menues and slighly changing the node.
	\subsection{Parameters}
		We will, on the same computer, register a new device in The Things Network, change the node files to include its new authentication codes and upload them to the Arduino.
	\subsection{Results}
		The first atempt took 4:50 and the second one 01:02.
	\subsection{Matching results and expectations}
		The first atempt took a bit longer than expected, perhaps because we didn't remember very well how to navigate the menus and options on The Things Network. On the second run we were even faster than predicted, maybe because what takes the most time is simply navigating the menus, and after knowing exacly what do it is very fast. This experiment is not a complete demonstration of the scalability of our system, but it is the best we could come up with given the resources.

\section{Throughput}

	\subsection{Pourpose}

	\subsection{Expectations}

	\subsection{Parameters}

	\subsection{Results}

	\subsection{Matching results and expectations}


%%% Local Variables:
%%% mode: latex
%%% TeX-master: "../ClassicThesis"
%%% ispell-dictionary: "british" ***
%%% mode:flyspell ***
%%% mode:auto-fill ***
%%% fill-column: 76 ***
%%% End:
