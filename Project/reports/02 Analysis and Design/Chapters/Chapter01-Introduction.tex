\chapter{Introduction}
\label{cha:introduction}

Recent advances in sensor technology have made it possible to embed sensors in everyday objects to collect data related to various parameters. With the advent of low power network technologies like LoRaWAN, it is possible to create a sensor network which requires little power and low maintenance.

Furthermore, current development in the field of machine learning and big data have made it possible to acquire valuable insights from the collected data and have thus made the acquisition of data even more lucrative.

Healthcare is a field of growing interest for this data collection, as the remote monitoring of patients not only allows us to reduce our response time for emergencies but also gives us valuable insights about the patient's health, which in turn leads to better diagnosis and insight about a disability or disease.

Our project consists of building a smart walker which leverages the low power networking capabilities of the LoRaWAN protocol, decentralized computing in the cloud and the power of data analysis to monitor and analyze the health parameters of it's user.

More concretely, we aim to track usage statistics, the position and common usage area of walkers, and monitor some health parameters of the user.

The objective of our project is thus to test the following hypothesis:
"Is it practical to monitor walkers and their users by sending data over LoRaWAN?". We define a system to be practical if it has the following properties:

\begin{enumerate}
	\item Feasibility: We define a system to be feasible if it is possible to build a system in a given time frame with finite resources.
	\item Consistency: We define a system to be consistent if it behaves the same way over a period of time in response to a given action.
	\item Accuracy: We defience a system to be accurate if its output has low deviation with respect to the true value of a given measurement.
	\item Reliability: We define a system to be reliable if it has a low probability of failure.
	\item Durability: We define a system to be reliable if it has a long operational life.
	\item Modularity: We define a system to be modular if it is relatively easy to add or remove components from it.
	\item Scalability: We define a system to be scalable based on how well it performs in case of increase in the number of users.
	\item Servicability: We define a system to be servicable if it relatively easy to diagnose and fix errors occuring in the system.
\end{enumerate}


%%% Local Variables:
%%% mode: latex
%%% TeX-master: "../ClassicThesis"
%%% ispell-dictionary: "british" ***
%%% mode:flyspell ***
%%% mode:auto-fill ***
%%% fill-column: 76 ***
%%% End:
