\chapter{Introduction}
\label{cha:introduction}

Recent advances in sensor technology have made it possible to embed sensors in everyday objects to collect data related to various parameters. With the advent of low power network technologies like LoRaWAN, it is possible to create a sensor network which requires little power and low maintenance.

Furthermore, current development in the field of machine learning and big data have made it possible to acquire valuable insights from the collected data and have thus made the acquisition of data even more lucrative.

Healthcare is a field of growing interest for this data collection, as the remote monitoring of patients not only allows us to reduce our response time for emergencies but also gives us valuable insights about the patient's health, which in turn leads to better diagnosis and insight about a disability or disease.

Our project consists of building a smart walker which leverages the low power networking capabilities of the LoRaWAN protocol, decentralized computing in the cloud and the power of data analysis to monitor and analyze the health parameters of it's user.

More concretely, we aim to track usage statistics, the position and common usage area of walkers, and monitor some health parameters of the user.

%%% Local Variables:
%%% mode: latex
%%% TeX-master: "../ClassicThesis"
%%% ispell-dictionary: "british" ***
%%% mode:flyspell ***
%%% mode:auto-fill ***
%%% fill-column: 76 ***
%%% End:
